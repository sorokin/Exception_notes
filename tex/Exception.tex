\section{Exceptions}
\subsection{Введение}
В ходе написания программ, часто возникают случаи, когда нам приходится отвлекаться от написания основной логики. Нам нужно учитывать граничные случаи и ошибки, которые могут возникнуть как во внутреннем коде нашей программы, так и во внешнем коде. Это может быть: нехватка динамической памяти, неправильный ввод пользователя, ошибка с файловой системой и т. д.
Все это приходится продумывать, и иногда это может сильно увеличить количество кода и время его написания. Чтобы упростить обработку таких случаев существует паттерн исключений.

Исключение - это ситуация, которую мы сочли исключительной, и которую хотим отделить от основной логики программы. Естественно код обработки исключения мы тоже должны написать сами. Но компилятор помогает нам, генерировать дополнительную информации об исключении, отслеживать и подбирать необходимый код обработки в зависимости от вида исключения.

Пусть мы пишем функцию деления:
\begin{minted}[linenos, frame=lines, framesep=2mm, tabsize = 4, breaklines]{c++}
void div(int a, int b) {
    return a / b;
}
\end{minted}

Пусть исключительная ситуация возникает тогда, когда $b = 0$. И мы хотим сообщить об этом пользователю нашей функции.

\begin{minted}[linenos, frame=lines, framesep=2mm, tabsize = 4, breaklines]{c++}
class Division_by_zero() {
    int dividend
    string message;
    Division_by_zero(int &dividend, string const &message) :
        dividend(dividend), message(message) { }
};
\end{minted}

Мы объявили класс, объекты которого будут хранить в себе всю информации об исключении. Тип класса-исключения называется просто типом исключения.

\begin{minted}[linenos, frame=lines, framesep=2mm, tabsize = 4, breaklines]{c++}
void div(int a, int b) {
    if (b == 0) // возникает исключительное состояние
        throw Division_by_zero(a, "in function div(int, int)"); // генерируем исключение.
    return a / b;
}
\end{minted}

Если возникает ситуация, когда нам передали в качестве делителя ноль, то мы создаем/генерируем/возбуждаем исключение. Это значит, что дальше не будет исполняться основная логика программы, пока не будет обработано исключение.

А так пользователь может обрабатывать исключение:
\begin{minted}[linenos, frame=lines, framesep=2mm, tabsize = 4, breaklines]{c++}
int main() {
    int n;
    cin >> n;
    try { // здесь указываем опрераторы, в которых мы хотим ловить исключения.
        for (int i = 0, a, b; i < n; ++i) {
            cin >> a >> b;
            cout << div(a, b);
        }
    } catch(Division_by_zero obj) { //здесь указываем тип исключения, которое мы хотим обработать
    // здесь обрабатываем исключение
        cout << obj.dividend << "div by 0 " << obj.message();
    }
}
\end{minted}

Как только возникает попытка деления на ноль. генерируется исключение, в коде \mintinline{c++}{div(int, int)} оно не обрабатывается, поэтому выбрасывается во внешний код, где мы его ловим и выводим сообщение об ошибке.

Если исключения не возникает, то код отработает нужным образом: мы выведем результаты всех делений.

\subsection{Описание конструкций}
Теперь рассмотрим используемые здесь конструкции подробнее:

\mintinline{c++}{try{}} -- защищенный блок. Здесь мы пишем код, в котором мы хотим ловить и обрабатывать исключения.

\mintinline{c++}{catch(){}} -- блок перехвата исключений или блок обработки или обработчик. Здесь будут ловиться исключения, тип которых совпадает по определенным правилам с типом указанным в (), и обрабатываться инструкциями в \{\}

\mintinline{c++}{throw} -- оператор инициализации исключения. Он генерирует исключение. (Обычно говорят, "бросает"\ или "выбрасывает"\ исключение)

Теперь давай рассмотрим детали работы этого механизма.

Блок \mintinline{c++}{try-catch} - реализует обработку исключений. И имеет общий вид:
\begin{minted}[linenos, frame=lines, framesep=2mm, tabsize = 4, breaklines]{c++}
try { /*операторы защищенного блока*/ }
// catch-блоки
catch() {/*код обработки*/}
...
catch() {/*код обработки*/}

\end{minted}
То есть обработчиков может быть несколько, каждый обрабатывает свой тип исключений.

Общий вид catch-блока:
\begin{itemize}
    \item
    \mintinline{c++}{catch(Type) { /*обработчик исключения*/ }}
    Мы не используем генерируемый исключением объект.
    \item
    \mintinline{c++}{catch(/*declaration exception_variable*/) { /*обработчик исключения*/ }} Теперь мы можем использовать объект исключения. Например, вывести информацию об исключении.
    \item
    \mintinline{c++}{catch(...) { /*обработчик исключения*/ }}
    Мы ловим исключение с любым типом, но не можем получить к нему доступ.
\end{itemize}

Что происходит, когда мы генерируем исключение:
\begin{enumerate}
    \item
    Создается копия объекта переданного в оператор throw. Этот объект будет существовать до тех пор, пока исключение не будет обработано. Если тип объекта имеет конструктор копирования, то он будет вызван.
    \item
    Прерывается выполнение защищенного try-блока.
    \item
    Выполняется раскрутка стека, пока исключение не будет обработано.
\end{enumerate}

При раскрутке стека, вызываются деструкторы локальных переменных в обратном порядке их объявления. При разрушении всех локальных объектов текущей функции процесс продолжается в вызывающей функции. Раскрутка стека продолжается пока не будет найдем try-catch -блок. При нахождении try-catch-блока, проверяется, может ли исключение быть поймано одним их catch-блоков.

\subsection{Как ловится исключение?}

Catch-блоки проверяются в том порядке, в котором написаны. Обработчик считается подходящим если:
\begin{enumerate}
    \item
    Тип, указанный в catch-блоке, совпадает с типом исключения или является ссылкой на этот тип.
    \item
    Класс, заданный в catch-блоке, является предком класса, заданного в throw, и наследование открытое (public).
    \item
    Указатель, заданный в операторе throw, может быть преобразован по стандартным правилам к указателю, заданному в catch-блоке.
    \item
    В catch-блоке указанно многоточие.
\end{enumerate}

Если найдет нужный catch-блок, то выполняется его код, остальные catch-блоки игнорируются, а выполнение продолжается после try...catch-блока и исключение считается обработанным. Если ни один catch-блок не подошел, процесс раскрутки стека продолжается.

\textcolor{red}{NB}) Так как поиск ведется последовательно, то нужно учитывать порядок catch-блоков (Например, catch(...) должен быть последним).

В некоторых случаях внутри catch-блока может быть необходимо не завершать раскрутку стека. Для этого существует специальная форма оператора throw без аргумента. Она означает проброс текущего исключения. Исключение все еще считается не обработанным.

\textcolor{red}{NB}) При это следует заметить, что при повторном выбросе исключения рассматривается не параметр текущего catch-блока, а именно изначальный статический объект. Именно он скопируется в качестве параметра в следующий обработчик. Поэтому его этот статический объект живет пока его исключение не обработается полностью. Поэтому при приведении тип исключения и данные внутри объекта не теряются.


Пример:
\begin{minted}[linenos, frame=lines, framesep=2mm, tabsize = 4, breaklines]{c++}
class Exception {
public:
    Exception(): value(0) { }
    int value;
};

void third() {
    throw Exception();
    std::cout << "end third\n";
}

void second() {
    try {
        third();
    }
    catch (Exception exc) {
        std::cout << "in second Exception-value = " << exc.value << std::endl;
        exc.value = 100;
        throw;
    }
    std::cout << "end second\n";
}

void first() {
    try {
        second();
    }
    catch (Exception exc) {
        std::cout << "in first Exception-value = " << exc.value << std::endl;
    }
    std::cout << "end first\n";
}

int main() {
    first();
    std::cout << "end main\n";
    return 0;
}
\end{minted}

\textbf{Вывод программы:} \\
in second() Exception-value = 0 \\
in first() Exception-value = 0 \\
end first \\
end main \\

\textcolor{red}{NB}) Также при наследовании классов исключений следует различать catch(type obj) и catch(type\& obj). В первом случае при входе в catch блок делается копия объекта-исключения. Во втором случае obj лишь ссылается на этот объект и копии не создается.

Пример:
\begin{minted}[linenos, frame=lines, framesep=2mm, tabsize = 4, breaklines]{c++}
struct Exception_base {
    virtual char const* msg() const {
        return "base";
    }
};

struct Exception_derived : Exception_base {
    virtual char const* msg() const {
        return "derived";
    }
};

int f() {
    try {
        throw derived();
    }
    catch (base e) {
        std::cout << e.msg() << std::endl;
    }
}

int g() {
    try {
        throw derived();
    }
    catch (base const& e) {
        std::cout << e.msg() << std::endl;
    }
}
\end{minted}

В данном примере g() выводит <<derived>>, а функция f() выводит <<base>>, поскольку объект исключения был скопирован с базы объекта, который мы передали в оператор throw и новая копия имеет тип base.

\subsection{Function-try-block}

Часто мы хотим, чтобы все тело функции находилось в try-блоке. Тогда это try-блок называется функциональным. И для него есть отдельный синтаксис.
\begin{minted}[linenos, frame=lines, framesep=2mm, tabsize = 4, breaklines]{c++}
int main() try {
    //main's body
}
catch (...) { }
\end{minted}

Здесь функциональные try-блоки являются синтаксическим сахаром, но есть ситуации когда без них не обойтись: обработка исключений в конструкторе.

Вот есть класс, котором мы хотим ловить и обрабатывать исключения.
\begin{minted}[linenos, frame=lines, framesep=2mm, tabsize = 4, breaklines]{c++}
class St {
public:
    St(): member() {
        try {
            // Constructor's code
        }
        catch (...) { }
    }
private:
    Member_type member;
}
\end{minted}

Но заметим, что вызов конструкторов членов не находится внутри try-блока и исключения возникшие в их конструкторах не поймаются.
Поэтому мы используем здесь функциональный try-блок:

\begin{minted}[linenos, frame=lines, framesep=2mm, tabsize = 4, breaklines]{c++}
class St {
public:
    St() try: member() {
        // Constructor's code
    }
    catch (...) {

    } // implicit throw
}
private:
    Member_type member;
}
\end{minted}

Но у функциональный try-блок в конструкторах, есть особенность: они всегда бросают исключение повторно.

\subsection{Уничтожение объекта при исключении в конструкторе}

Также важно помнить, что если в конструкторе происходит исключение, то для него не вызовется деструктор, так как объект еще не считается созданным. Захваченные ресурсы придется очищать руками

Важно понимать, что происходить в конструкторе когда возникает исключение. Если исключение возникает при созднии члена класса, то от всех уже созданных членов вызываются деструкторы. Но от самого объекта деструктор не вызвается, так как объект не считается созданным пока хотя бы один конструктор не отработал полностью. Поэтому при необходимости нужно либо руками освобождать захваченные ресурсы, либо декларировать небросающему конструктору.

\subsection{Best practice}
Часто исключения применяются для корректной работы с ресурсами. То есть если возникает исключение и мы владеем какими-то ресурсам, то в случае генерации исключения следует их освободить.

Например, мы пишем конструктор копирования для вектора, и нам необходимо скопировать данные в другой участок памяти. При этом если во время копирование какого-то объекта возникнет исключение, то хорошо если уже созданные объекты будут разрушены. Причем желательно в обратном порядке их создания.
\begin{minted}[linenos, frame=lines, framesep=2mm, tabsize = 4, breaklines]{c++}
template <typename T>
void copy_construct(T* dist, T const * sourse, size_t size) {
    size_t i = 0;
    try {
        for (; i != size; ++i) {
            new (dist + i) T(sourse[i]);
        }
    }
    catch (...) { // если ошибка при копировании
        for (size_t j = i; j != 0; --j) {
            dist[j - 1].~T(); // вызовем деструкторы созданных объектов
        }
        throw;
    }
}
\end{minted}

Полезно знать про стандартные исключения, такие как  \mintinline{c++}{std::bad_alloc}, \mintinline{c++}{std::bad_cast}, \mintinline{c++}{std::bad_typeid} и т. д. Они также связанны наследованием и имеют общего предка \mintinline{c++}{std::exception}.

Подробнее можно почитать здесь: \\
\url{https://www.tutorialspoint.com/cplusplus/cpp_exceptions_handling.htm} \\
\url{http://en.cppreference.com/w/cpp/error/exception} \\

Хорошим тоном является наследование от \mintinline{c++}{std::exception}.

Когда мы организовываем исключения в иерархии классов, то получаем мощный механизм описания исключение и способов их обработки. Создав такую структуры мы можем обрабатывать как более общие ошибки, так и более специализированные.
Пример:
\begin{minted}[linenos, frame=lines, framesep=2mm, tabsize = 4, breaklines]{c++}
    class StackException {};
    class popOnEmpty(): public StackException {};
    class pushOnFull(): public StackException {};
\end{minted}
	Причем сгенерировав исключение типа popOnEmpty(), мы можем в разных обработчиках независимо выбирать: обработать как popOnEmpty или как StackException, так как тип исключения не теряется при повторной генерации этого исключения.

\subsection{Bad practice}
\begin{itemize}
\item
Писать код бросающий исключения в catch-блоке. Это не всегда бывает просто.
\item
Хотя если исключений не происходит, то по скорости выполнения программа не сильно упадет, но иногда нужно учитывать большой overhead в случае возникновения и обработки исключения.
\end{itemize}

\subsection{std::terminate()}

Это функция, которая вызывается если у механизма исключений не получается корректно отработать, чтобы завершить программу.
Случаи когда она вызывается:
\begin{itemize}
\item Исключение брошено и не поймано ни одним catch-блоком, то есть пробрасывается вне main().
\item Исключение бросается во время обработки другого исключения. Это может произойти только в catch-блоке или деструкторе. А также в функциях, которые вызываются ими.
\item Если функция переданная в \mintinline{c++}{std::atexit} и \mintinline{c++}{std::at_quick_exit} бросит исключение.
\item Если функция нарушит гарантии noexcept specification. Например, если функция помеченная как noexcept бросит исключение.
\item При подобных и не только ошибках в потоках.
\end{itemize}

По умолчанию \mintinline{c++}{std::terminate()} вызвает \mintinline{c++}{std::abort()}, но можно это изменить, написав свою функцию \mintinline{c++}{my_terminate()} и зарегистрировать ее как терминальную.
\begin{minted}[linenos, frame=lines, framesep=2mm, tabsize = 4, breaklines]{c++}
void my_terminate() {
    cout << "It's not a bug, it's a feature!";
}
/**/
set_terminate(my_teminate);
\end{minted}
